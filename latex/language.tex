\section{Language}
\label{sec:featurecompare:language}

GeoMesa is developed using Scala, and GeoWave is developed using Java.
Both Scala and Java are languages in which the source compiles down to Java Virtual Machine (JVM) bytecode, which is executed on top of the same JVM.
This means that both projects can use the same dependencies, as can be observed by each project's reliance on GeoTools (a Java based geospatial library) for some of its features, including a core data type: the GeoTools \texttt{SimpleFeature}.
However, the differences between the Scala and Java languages are many, and it remains one of the biggest differences between the projects.

%% And user familiarity with the implementation languages should be considered as factor.
%% We will address this from the perspective of Java developers as they are a larger group of potential users.
%% GeoWave uses standard and well structured Java composition and design patterns both in its API and implementation and should provide few surprises for Java developers.
%% GeoMesa, though written in Scala, implements a GeoTools API that allows Java developers to easily use GeoMesa functionality without having to write Scala.
%% Java interoperability is a core design concern in GeoMesa API as evidenced by tutorial code being provided exclusively in Java.
%% This has limits, however, as developers investigating stack traces and exploring the GeoMesa implementation will encounter Scala.
%% Being Scala developers we can comment that GeoMesa codebase uses direct dialect of Scala, eschewing advanced language features and staying close to Java patterns which optimizes its readability for such a developer.
%% Ultimately the feasibility of this should be part of individual team evaluation when adopting either project.
