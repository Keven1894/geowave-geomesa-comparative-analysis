\section{Discussion}
\label{sec:subjective}



While working with the GeoMesa and GeoWave projects, we were able to get a sense of the ``open source maturity'' of the projects.
This open source (and not technical) maturity can be important in many ways, for instance increasing the chance that beginners can use the project successfully.
Although open source maturity is a subjective assessment, in working with these two projects, we found GeoMesa to be the more mature project based on a few important factors.
For instance, documentation is a significant part of being a usable open source project.
GeoMesa documentation is more clear and covers more specific questions arising from attempting to use the product.
There are also more clear ways to do things in GeoMesa, such as using the command line tooling to convert and ingest various data schemas in a distributed fashion.
Also, we experienced fewer instances of unexpected behavior when using less common feature configurations in GeoMesa.

It's important to note that this opinion does not reflect the capabilities of the projects, but rather the resources the project has committed to specific aspects of being a project in the open source ecosystem.
The conclusion we've made makes sense given the history of the projects in the open source: GeoWave was open sourced after GeoMesa, and has had a large focus on use cases internal to the NGA.
GeoMesa graduated as a top level LocationTech project.
GeoWave has not yet started incubation due to a number of factors outside the project's control.
We are confident that GeoWave will continue to gain maturity by addressing issues discussed in this report and by a growing user base.
