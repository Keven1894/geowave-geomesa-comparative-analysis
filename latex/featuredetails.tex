\section{Details of GeoMesa and GeoWave features}
\label{appendix:features}

Note: This appendix refers to features in GeoMesa 1.2.6 and GeoWave 0.9.2.

\subsection{GeoMesa Feature List}
\label{appendix:features:geomesa}

\subsubsection*{Data Ingest/Input}

\begin{itemize}
\item{Command line tools for interacting with GeoMesa, which provides the ability to:
  \begin{itemize}
  \item Create a GeoMesa datastore for Accumulo
  \item{Ingest \texttt{SimpleFeature} data
    \begin{itemize}
    \item Predefined, common \texttt{SimpleFeatureType}s are provided - GDELT, Geolife, geonames, gtd, nyctaxi, osm-gpx, tdrive, twitter
    \end{itemize}
  }
  \item{Ingest rasters
    \begin{itemize}
    \item Supported file formats: ``tif'', ``tiff'', ``geotiff'', ``dt0'', ``dt1'', ``dt2''
    \item Note: Raster support is limited; e.g. you are required to have the rasters pre-tiled, and they must be in the EPSG:4326 projection.
    \end{itemize}
  }
  \end{itemize}
}
\item{Tools for converting various serialization formats to \texttt{SimpleFeature}s for ingest:
  \begin{itemize}
  \item{Conversion mechanisms are specified by way of configuration files.  Supported formats include:
    \begin{itemize}
    \item Delimited text: CSV, DEFAULT, EXCEL, MYSQL, TDF, TSV, RFC4180, QUOTED, QUOTE\_ESCAPE, QUOTED\_WITH\_QUOTE\_ESCAPE
    \item Fixed Width
    \item Avro
    \item JSON
    \item XML
    \end{itemize}
  }
  \end{itemize}
}
\item{Support for streaming input:
  \begin{itemize}
  \item A datastore which listens for updates from a supported streaming source
  \item Generic apache-camel based implementation of a streaming source
  \item Hooks for updating GeoServer on stream update
  \end{itemize}
}
\item Storm/Kafka ingest (mentioned in ``Other Features'' below)
\end{itemize}


\subsubsection*{Data Processing}

\begin{itemize}
\item{Spark integration
  \begin{itemize}
  \item Generating RDDs of \texttt{SimpleFeature}s
  \item Initial support for carrying out Spark \texttt{SQL} queries to process GeoMesa data
  \end{itemize}
}
\item{Hadoop integration
  \begin{itemize}
    \item Reading data for use in a custom MapReduce job
  \end{itemize}
}
\item{Processing on Accumulo backed GeoMesa instances
  \begin{itemize}
  \item Computing a heatmap from a provided \texttt{CQL} query
  \item {
    Computing statistics from a \texttt{CQL} query:
    \begin{itemize}
    \item Currently supported statistics: count, enumeration, frequency (countMinSketch), histogram, top-$k$, and min/max (bounds).
    \item Command line tools expose the following statistics: count, histogram, min/max (bounds), and top-$k$.
    \end{itemize}
    }
  \item{
    ``Tube selection'' (space/time correlated queries): This is a pretty sophisticated query mechanism.
    The basic idea is that, given a collection of points (with associated times),
    you should be able to return similar collections of points (in terms of where the lines connecting said points exist).
    Constraints on the query include the size of the spatial and temporal buffers (this is the sense in which we're dealing with 'tubes') and maximum speed attained by the entity whose points make up a given trajectory.
    Read more here: \url{http://www.geomesa.org/documentation/tutorials/geomesa-tubeselect.html}.
  }
  \item{
    Proximity Search: Given a set of vectors to search through and a set of vectors to establish proximity,
    return the members of the former set which lie within the (specified) proximity of members of the latter set.
  }
  \item Queries take advantage of Accumulo optimization to carry out GeoMesa queries
  \item Find the $k$-nearest neighbors to a given point
  \item Identify unique values for an attribute in results of a \texttt{CQL} query
  \item{
    Convert points to lines: Convert a collection of points into a collection of line segments given a middle term parameter.
    Optionally break on the day of occurrence. This feature isn't really advertised.
  }
  \end{itemize}
}
\end{itemize}

  
\subsubsection*{Indices}

\begin{itemize}
\item{Default Indices
  \begin{itemize}
  \item{
    {\bf XZ3.}
    This is the default for 3D objects with extent in GeoMesa 1.2.5.
    Objects are indexed with a maximum resolution of $36$ bits ($12$ divisions) into eighths.
  }
  \item{
    {\bf XZ2.}
    This is the default for 2D objects with extent in GeoMesa 1.2.5.
    Objects are indexed with a maximum resolution of $24$ bits ($12$ divisions) into quarters.
  }
  \item{
    {\bf Z3.}
    This is used for points; $x$, $y$, and $t$ (time) have resolutions of $21$, $21$, and $20$ bits, respectively.
  }
  \item{
    {\bf Z2.}
    This is used for points; $x$ and $y$ both have resolutions of $31$ bits.
  }
  \item{
    {\bf Record.}
    This is an index over object \texttt{UUID}s.
  }
  \end{itemize}
}
\item{Optional Indices
  \begin{itemize}
  \item{
    {\bf Attribute.}
    This is an index over \texttt{SimpleFeature} attributes.
    Allows one to create a join index over the \texttt{UUID}, date, and geometry.
  }
  \item{{\bf ST.}  This Spatio-Temporal Index is deprecated.}
  \end{itemize}
}
\item Cost-Based Optimization (CBO): Used to select which index to use for a query when data have been ingested with multiple indexes.
\end{itemize}


\subsubsection*{Output}

\begin{itemize}
\item{
  Accumulo output:
  \begin{itemize}
  \item A reader for directly querying a datastore in java/scala
  \item Direct map/reduce exports
  \end{itemize}
}
\item{
  Command line tools for interacting with GeoMesa:
  \begin{itemize}
  \item Serialize and export stored features (vectors).  Supported export formats include : CSV, shapefile, geojson, GML, BIN, and Avro.
  \end{itemize}
}
\item The ability to return only a subset of \texttt{SimpleFeature} attributes, reducing the size of return values.
\end{itemize}


\subsubsection*{Other Features}

\begin{itemize}
\item{
  GeoMesa Native API.
  This is an alternative to the geotools interface for interaction with GeoMesa stores.
}
\item HBase backend
\item Google BigTable backend
\item BLOB backend
\item Sampling of data for custom statistics
\item Cassandra backend (alpha quality)
\item A Kafka geotools datastore to pipe simplefeature types from producers, through kafka, to consumers
\item{
  Metrics reporting.
  This offers real time reporting of performance for GeoMesa instances.
  Supports multiple reporting backends - Ganglia, Graphite, and CSV/TSV.
}
\end{itemize}

\subsection{GeoWave Feature List}
\label{appendix:features:geowave}

\subsubsection*{Input}

\begin{itemize}
\item{
  Ingest from the CLI
  \begin{itemize}
  \item Ingest from filesystem $\rightarrow$ GeoWave
  \item Ingest from filesystem $\rightarrow$ HDFS $\rightarrow$ GeoWave
  \item Stage from filesystem $\rightarrow$ HDFS
  \item Stage from filesystem $\rightarrow$ Kafka
  \item Ingest from Kafka $\rightarrow$ GeoWave
  \item Ingest from HDFS $\rightarrow$ GeoWave
  \item Notes: Requires plugins for each input file format, which are listed below in "File Formats Supported"
  \end{itemize}
}
\item{
  Ingest Using the API
  \begin{itemize}
  \item Bulk ingest via Hadoop
  \item Piecemeal via a writer object
  \end{itemize}
}
\item{
  File Formats Supported
  \begin{itemize}
  \item avro
  \item gdelt
  \item geolife
  \item geotools-raster (GeoTools-supported raster data)
  \item geotools-vector (GeoTools-supported vector data)
  \item gpx
  \item stanag4676
  \item tdrive
  \item{
    Formats supported via Extensions existing outside the GeoWave repository:
    \begin{itemize}
    \item Landsat 8
    \item OpenStreetMap
    \end{itemize}
  }
  \end{itemize}
}
\end{itemize}


\subsubsection*{Backends}

\begin{itemize}
\item Accumulo
\item HBase
\end{itemize}

\subsubsection*{Integrations}

\begin{itemize}
\item MrGeo (reading)
\item GeoTrellis - (raster and vector, reading and writing)
\item{
  Via C++ bindings
  \begin{itemize}
  \item PDAL (reading and writing)
  \item mapnik (reading)
  \end{itemize}
}
\end{itemize}


\subsubsection*{Secondary Indices}

\begin{itemize}
\item Numerical
\item Temporal
\item Textual
\item User Defined
\item See section on comparision of secondary indexing below for more details
\end{itemize}

\subsubsection*{Processing}

\begin{itemize}
\item $k$-means via CLI or Map-Reduce
\item Jump Method ($k$-discovery) via CLI and Map-Reduce
\item Sampling via Map-Reduce
\item Kernel Density Estimation via CLI or MapReduce
\item Nearest Neighbors via CLI or MapReduce
\item Clustering via Map-Reduce
\item Convex Hulls of Clusters via Map-Reduce
\item \texttt{DBSCAN} via Map-Reduce
\item Spark Support: Ability to load an RDD of SimpleFeatures
\end{itemize}

\subsubsection*{Output}

\begin{itemize}
\item{
  GeoServer Plugin
  \begin{itemize}
  \item Includes the "decimation" feature, which allows large datasets to be shown interactively by subsampling at the pixel level.
  \end{itemize}
}
\item Hadoop integration
\item{
  Query
  \begin{itemize}
    \item GeoWave DataStore: directly construct queries via the GeoWave API
    \item GeoTools DataStore: construct queries via CQL
    \item Ability to query data that has start and end times to find intersecting time intervals.
  \end{itemize}
}
\end{itemize}


\subsection{Comparison of Attribute/Secondary Indices Feature}
\label{appendix:features:indices}

Often, spatial coordinates aren't the only important condition used in searching for and filtering through a dataset.
Paramedics might want to find only accidents within their geographic region but they also might only want those accidents whose ``severity'' attribute is ``fatal''.
For certain applications it is a matter of practical necessity that such fields are indexed for quick lookup later and both GeoMesa and GeoWave provide some tools for these purposes.
It is worth mentioning that the documentation provided by both projects suggests that secondary/attribute indices are an area that will receive future focus by their respective teams.
In what follows, we briefly compare the features provided by each.


\subsubsection{GeoMesa Attribute Indices}

In GeoMesa, any attribute can be indexed with a simple modification to the UserData which is associated with a SimpleFeatureType's attribute.
Each attribute index is stored in a single, associated \texttt{attr\_idx} table.
By fiat, let's imagine we have a SimpleFeatureType which describes car accidents as described above.
The following code will add the appropriate information to our type so that, upon ingest, indices are created to the values in our ``severity'' field:

\begin{algorithm}[htb]
\caption{GeoMesa attribute indexing code snippet.}
\footnotesize{\begin{lstlisting}
val sft: SimpleFeatureType = ??? // Our feature's schema
sft.getDescriptor("severity").getUserData().put("index", "join");
sft.getDescriptor("severity").getUserData().put("cardinality", "high");
\end{lstlisting}}
\end{algorithm}

As seen above, two properties on this attribute index are exposed through the UserData : ``index'' (the type of index operation) and ``cardinality'' (the number of distinct values).

{\bf Full/Join Indices.}
This type of index - ``full'' or ``join'' - determines how much data is replicated in the lookup table of the attribute index.
Full indices store the entire SimpleFeature of a record, allowing for quick replies to indexed-attribute queries without joining against the records table.
This is preferable under circumstances in which the attribute in question is regularly queried against and especially if the expected queries don't necessarily rely upon other fields for filtration.
The ``join'' index stores only the data necessary for identifying the values in the records table which satisfy the provided predicate and is therefore useful for preserving storage resources.

{\bf Low/High Index Cardinality.}
The utility of this distinction is somewhat unclear.
A high cardinality index has enough values that we can expect any filtering it does to significantly slim down the number of returned records (thus, a query against a high cardinality index is given priority) while a low cardinality index seems to be ignored.
The user documentation under 'Data Management' notes (as of 10/01/2016) that ``technically you may also specify attributes as low-cardinality - but in that case it is better to just not index the attribute at all.''

{\bf Client Code Difficulties.}
As of 1.2.6, it appears as though a library which is shaded in GeoMesa client code needs to be appropriately shaded in any ingest client code which intends to take advantage of attribute indices.
The fix for this issue can be found in a commit which made its way into 1.2.6.

\subsubsection{GeoWave Secondary Indices}
  
Unlike GeoMesa, each secondary index gets its own table. Unlike GeoMesa, setting these secondary indices up is not a simple, two-line affair. Figuring out how to actually use these secondary indices was not obvious or straightforward from the documentation.

Here we modify the same SimpleFeatureType for extra indexing on ingest as above:

\begin{algorithm}[htb]
\caption{GeoWave secondary indexing code snippet.}
\footnotesize{\begin{lstlisting}
val sft: SimpleFeatureType = ???
val secondaryIndexingConfigs = mutable.ArrayBuffer[SimpleFeatureUserDataConfiguration]()
val textFieldsToIndex = Set("severity")

secondaryIndexingConfigs += new TextSecondaryIndexConfiguration(textFieldsToIndex.asJava)
val config = new SimpleFeatureUserDataConfigurationSet(sft, secondaryIndexingConfigs.asJava)
config.updateType(sft)
\end{lstlisting}}
\end{algorithm}

{\bf Index Cardinality.}
Unlike GeoMesa, cardinality of indices isn't a static feature configured by the user.
GeoWave's query planning and optimization attempts to determine the usefulness of an index for a given query based on the statistics it gathers on ingest.

{\bf Specialized Index Types.}
Another point of divergence between these projects in terms of extra index support is GeoWave's intent to support specialized indices which can take advantage of various assumptions which are domain specific.
Exact-match (as opposed to fuzzy) indices for text are not the same as exact indices for numbers or dates or even fuzzy indexing (through n-grams) of that same text.
The specialization here makes it possible for GeoWave to index in ways that are sensitive to the types of data in question and even to the expectations of use (i.e. fuzzy vs exact and range-based vs exact queries).

{\bf Future Development.}
Documentation for GeoWave mentions the possibility of adding n-gram based fuzzy indexing of text fields (so that searches based on a subset of the data in a field can be used).
It appears as though this feature is already in the works, as an n-gram table is currently generated on ingest in the development branch of GeoWave.
