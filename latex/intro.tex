\section{Introduction}
\label{sec:introduction}

GeoMesa and GeoWave are two open source projects that deal with large geospatial data.
At a high level, these projects have solutions to many of the same types of problems.
Because of this overlap, it has been difficult for new users approaching the big geospatial data community to understand what the differences are between these projects and what project should be used under what circumstances.
For some of their most overlapping functionality, for example, indexing spatial and spatiotemporal data in Accumulo, the differences between the two projects can be unclear even to veterans of big geospatial data processing.
This document aims to address this lack of clarity.

In the summer of 2016, Azavea conducted a comparative analysis of GeoWave and GeoMesa in order to gain a deeper understanding of the two projects and to share that understanding with the geospatial big data community.
This document contains the results of our efforts and aims to provide a more clear picture of how the projects are different from each other and what use cases fit best to either project.
We hope this will aid the geospatial big data community in gaining a deeper understanding of these two outstanding projects and allow a better utilization of their functionality.

Along with an understanding how the projects are different from each other, this comparative analysis aims to provide information and guidance to potential future collaboration efforts between the GeoWave and GeoMesa projects.

This document assumes prior knowledge about the GeoMesa and GeoWave project and is not intended to be an introduction to those projects.
For background information, please see the project websites:


\begin{itemize}
\item  GeoWave: \url{http://ngageoint.github.io/geowave/}
\item  GeoMesa: \url{http://www.geomesa.org/}
\end{itemize}
