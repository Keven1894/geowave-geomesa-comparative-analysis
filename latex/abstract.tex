\begin{abstract}
  This document details the results of a comparative analysis between two open source geospatial big data frameworks: GeoWave and GeoMesa.
  A feature comparison and a set of performance tests with analysis are presented.
  We have concluded that despite a large set of overlapping features, specifically the capability to index and query spatial and spatiotemporal data in Accumulo, the projects differ from each other in substantial ways.
  Through analyzing performance test data, we make four conclusions about the performance characteristics of the current versions of the systems - GeoMesa 1.2.6 and GeoWave 0.9.3 - for the use case of indexing spatial and spatiotemporal data in Accumulo: (1) GeoMesa performed better against queries with large result counts - i.e. queries that were not highly selective - while GeoWave performed better on smaller result sets - i.e. queries that selected fewer results out of a larger dataset; (2) GeoWave performed better against queries with larger temporal bounds, while GeoMesa performed better when the temporal bounds were smaller (around a couple of weeks or less); (3) GeoMesa performed better in the non-point dataset use case; and (4) GeoWave outperformed GeoMesa in multitenancy use cases, where there are 16 to 32 queries being executed against the system in parallel.
  We also find the two systems perform reasonably well in all cases, and that neither system was dominant in performance characteristics.
  We conclude by providing recommendations for ways the two projects can collaborate moving forward in light of this analysis.
\end{abstract}
